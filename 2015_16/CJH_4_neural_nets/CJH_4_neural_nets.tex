%CJH_4_neural_nets.tex
%notes for the course origins COMS12303 taught at the University of Bristol
%2015 Conor Houghton conor.houghton@bristol.ac.uk

%To the extent possible under law, the author has dedicated all copyright 
%and related and neighboring rights to these notes to the public domain 
%worldwide. These notes are distributed without any warranty. 

\documentclass[11pt,a4paper]{scrartcl}
\typearea{12}
\usepackage{graphicx}
\usepackage{pstricks}
\usepackage{listings}
\usepackage{color}
\lstset{language=C}
\pagestyle{headings}
\markright{COMS12303 - Neural Nets - Conor}
\begin{document}

\subsection*{1. McCulloch Pitts Neurons}

The brain is made of neurons; these are cells which receive inputs
along fibers called denrites. These inputs drive complex
electrodynamics in the cell body, called the soma; if the voltage in
the soma is increase enough by the inputs it causes a spike of voltage
called an action potential, or just spike, to travel along another
fiber from the soma called the axon. The axon, in turn, carries this
signal on to other neurons. 

The junction from axon to dendrite is somewhat complicated, they are
joined by a synapse, a complex biological machine. When a spike
arrives at a synapse along the axon of the pre-synaptic neuron it
causes a change in the voltage of the dendrite of the post-synaptic
neuron. These changes are the inputs which in turn effect the soma of
the post-synaptic neuron. Some neurons have excitatory synapses, which
increase the voltage in the soma of the post-synaptic, some have
inhibitory neurons, which decrease that voltage. The synapse has a
strength which governs the size of the effect it has; this strength
can change with time in a process called plasticity which is imagined
to support learning.

In the early 40s two scientists, cybernetists in the language of the
day, Warren McCulloch and Walter Pitts tried to describe these
dynamics according to a language of computation. McCulloch and Pitts
were interesting men, both were polymaths and Pitts was largely
self-taught and had been for a while destitute and homeless before his
talent was spotted. 

McCulloch and Pitts publish a simplified model of the neuron in
1943. They imagined that the soma performed a direct summation of its
current inputs, in fact, at its simplest, the voltage dynamics in the
soma cause it to perform a time-weighted integration of its
input. Furthermore they supposed that the cell is either in a low
state where it does not fire action potentials, or a high state where
it does; with the transition between these two states determined by
the level of input. In other words, they got rid of the action
potentials and imagined neurons as having in-active and active
states. Thus the state of a neuron $i$ is
\begin{equation}
x_i=H(r_i-\theta)
\end{equation}
where $H$ is the Heaviside function, it is one for $r_i>\theta$ and
zero otherwise. $r_i$ is the sum of the inputs
\begin{equation}
r_i=\sum_{j} w_{ij}x_j
\end{equation}
where $w_{ij}$ measures the strength of the connect between $x_i$ and
$x_j$; it is the proxy for the synapse strength.
\end{document}
