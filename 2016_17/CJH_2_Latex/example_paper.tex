\documentstyle{article}


\begin{document}

\title{Difficult problems with solutions} 

\author
{Conor Houghton,$^{1}$\footnote{To whom correspondence should be addressed; E-mail:  conor.houghton@bristol.ac.uk} Jane Doe$^{1}$ and Sheila Citizen$^{2}$\\
\\
\normalsize{$^{1}$Department of Computer Science, University of Bristol,}\\
\normalsize{Merchant Venturers Building, Woodland Road, Bristol, England.}\\
\normalsize{$^{2}$Department of Arithmetic, University of Reading,}\\
\normalsize{Writing Road, Reading, England.}\\
}

\date{}

\maketitle


\begin{abstract}
There are none
\end{abstract}


\section{Introduction}
There are easy problems \cite{BrainBrainBrain1902a}, hard ones \cite{LightKnight2007a} and problems whose solutions are even harder than the problem that prompted it \cite{Swift1729a}. You should avoid all hard problems \cite{LightKnight2007a,Swift1729a}.
\begin{equation}
e^{-x^2/2\pi}+1=y
\end{equation}
Here is another equation
\begin{equation}
\int_{-\infty}^\infty e^{-x^2}dx=\sqrt{\pi}
\end{equation}
Here is a fraction:
\begin{equation}
f(x)=\frac{x^2+x}{\log{x}+1}
\end{equation}
Here is another equation
\begin{equation}\label{set_inclusion_eq}
n^2+3n+1\in \Theta(n^2)
\end{equation}
As we saw in Eq.~\ref{set_inclusion_eq} we have a element in a set.
Here is a matrix
\begin{equation}
A=\left(\begin{array}{cc}1&2\\2&4\end{array}\right)
\end{equation}
Here is a piecewise function which we call $f(x)$, \textsl{in italics} as opposed to $in italics$.
\begin{equation}
f(x)=\left\{\begin{array}{cl}0&x<0\\1&\mbox{otherwise}\end{array}\right.
\end{equation}







\bibliographystyle{unsrt}   % this means that the order of references
			    % is dtermined by the order in which the
			    % \cite and \nocite commands appear
\bibliography{sample_bibliography}  % list here all the bibliographies that
			     % you need. 
\end{document}
